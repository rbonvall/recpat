% vim: set fileencoding=utf-8 encoding=utf-8 tw=100:
\documentclass[spanish]{article}
\usepackage{babel}
\usepackage[utf8]{inputenc}
\usepackage{fullpage}
\usepackage{url}
\usepackage{palatino}
\usepackage{amsmath}
\usepackage{amssymb}
\usepackage{graphicx}

\newcommand{\pregunta}{\textit}

\title{Reconocimiento de Patrones \\ Tarea 4}
\author{Roberto Bonvallet \\ \url {<rbonvall@gmail.com>}}
\date{Agosto de 2008}

\begin{document}
\maketitle

\section*{Pregunta 1}
\pregunta{
    Considere un conjunto de clases $\omega_i$, $i = 1, 2, \ldots, c$ y un conjunto de
    características representadas en el vector aleatorio $x$.  Sea $\omega_j$ tal que
    $P(w_j\vert x)\ge P(w_i\vert x)$ para todo $i = 1, 2, \ldots, c$.
}
\begin{enumerate}
    \item \pregunta{Muestre que $P(\omega_j\vert x) \ge 1/c$.}
    \item \pregunta{Muestre que el mínimo error de clasificación está acotado por $(c - 1)/c$.}
    \item \pregunta{Describa la situación para la cual la cota anterior es una igualdad.}
\end{enumerate}

\section*{Pregunta 2}
\pregunta{
}

\section*{Pregunta 3}
\pregunta{
}

\end{document}
