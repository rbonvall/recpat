% vim: set fileencoding=utf-8 encoding=utf-8 tw=100:
\documentclass[spanish]{article}
\usepackage{babel}
\usepackage[utf8]{inputenc}
\usepackage{fullpage}
\usepackage{url}
\usepackage{palatino}
\usepackage{amsmath}
\usepackage{amssymb}
\usepackage{graphicx}

\newcommand{\pregunta}{\textit}

\title{Reconocimiento de Patrones \\ Tarea 3}
\author{Roberto Bonvallet \\ \url {<rbonvall@gmail.com>}}
\date{Agosto de 2008}

\begin{document}
\maketitle

\section*{Pregunta 1}
\pregunta{
    Estudiar los conceptos de métrica, distancia y similaridad.
    ¿Son estos conceptos equivalentes?
    Justifique usando algunos ejemplos.
}

\section*{Pregunta 2}
\pregunta{
    Estudiar las diferentes linkage métricas para dos clústers:
    single link, complete link, centroid link, average group link.
}

\section*{Pregunta 3}
\pregunta{
    Proponer medidas de similaridad para las siguientes representaciones:
    \begin{itemize}
        \item datos cuantitativos
            (señales ECG, matrices, funciones, tensores, imágenes digitales, etc.),
        \item datos categóricos
            (documentos, libros, símbolos, jeroglíficos, etc.)
    \end{itemize}
}

\section*{Pregunta 4}
\pregunta{
    Proponer medidas de similaridad para mezclas de datos.
}

\end{document}
