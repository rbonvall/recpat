% vim: set fileencoding=utf-8 encoding=utf-8 tw=100:
\documentclass[spanish]{article}
\usepackage{babel}
\usepackage[utf8]{inputenc}
\usepackage{fullpage}
\usepackage{url}
\usepackage{palatino}
\usepackage{amsmath}
\usepackage{amssymb}
\usepackage{graphicx}

\newcommand{\pregunta}{\textit}
\newcommand{\card}[1]{\lvert#1\rvert}

\title{Reconocimiento de Patrones---Tarea 3}
\author{Roberto Bonvallet \\ \url {<rbonvall@gmail.com>}}
\date{Agosto de 2008}

\begin{document}
\maketitle

\section*{Pregunta 1}
\pregunta{
    Estudiar los conceptos de métrica, distancia y similaridad.
    ¿Son estos conceptos equivalentes?
    Justifique usando algunos ejemplos.
}

\section*{Pregunta 2}
\pregunta{Estudiar las diferentes linkage métricas para dos clústers:}
\begin{itemize}
    \item \pregunta{single link:} la distancia entre el par de elementos más cercano de los
        clústers:
        \[d(C_1, C_2) = \min_{C_1\times C_2}\bigl\{d(x_1, x_2)\bigr\}.\]
    \item \pregunta{complete link:} la distancia entre el par de elementos más lejanos de los
        clústers:
        \[d(C_1, C_2) = \max_{C_1\times C_2}\bigl\{d(x_1, x_2)\bigr\}.\]
    \item \pregunta{centroid link:} la distancia entre los centroides de los clústers:
        \[d(C_1, C_2) = d\left(\frac{1}{\card{C_1}}\sum_{C_1} x_1, \frac{1}{\card{C_2}}\sum_{C_2} x_2\right).\]
    \item \pregunta{average link:} el promedio de las distancias entre todos los pares de elementos
        de los clústers:
        \[d(C_1, C_2) = \frac{1}{\card{C_1}\card{C_2}}\sum_{C_1\times C_2} d(x_1, x_2).\]
\end{itemize}

\section*{Pregunta 3}
\pregunta{
    Proponer medidas de similaridad para las siguientes representaciones:
    \begin{itemize}
        \item datos cuantitativos
            (señales ECG, matrices, funciones, tensores, imágenes digitales, etc.),
        \item datos categóricos
            (documentos, libros, símbolos, jeroglíficos, etc.)
    \end{itemize}
}

\section*{Pregunta 4}
\pregunta{
    Proponer medidas de similaridad para mezclas de datos.
}

\end{document}
