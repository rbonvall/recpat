% vim: set fileencoding=utf-8 encoding=utf-8:
\documentclass[spanish]{article}
\usepackage{babel}
\usepackage[utf8]{inputenc}
\usepackage{fullpage}
\usepackage{url}
\usepackage{palatino}
\usepackage{amsmath}
\usepackage{amssymb}
\usepackage{graphicx}


\title{Reconocimiento de Patrones \\
    Discusión del artículo \textit{Similarity~Measures}}
\author{Roberto Bonvallet \\ \url {<rbonvall@gmail.com>}}
\date{Agosto de 2008}

\begin{document}
\maketitle

% set members for next example
\newcommand{\SMA}{\Box}
\newcommand{\SMB}{\spadesuit}
\newcommand{\SMC}{\heartsuit}
\newcommand{\SMD}{\clubsuit}
\newcommand{\SME}{\diamondsuit}

\begin{align*}
    A &= \{\phantom{\SMA,  \SMB,} \SMC,          \SMD,          \SME\}  \\
    B &= \{\phantom{\SMA,} \SMB,  \SMC,          \SMD\phantom{, \SME}\} \\
    C &=          \{\SMA,  \SMB,  \SMC\phantom{, \SMD,          \SME}\} \\
\end{align*}

\begin{figure}
 \centering
 \includegraphics[bb=0 0 492 146]{imagenes/cons-dom-trans-60.png}
 % cons-dom-trans-60.png: 615x182 pixel, 90dpi, 17.36x5.14 cm, bb=0 0 492 146
 \caption{%
    Ejemplos de las propiedades de un espacio con estructura de proximidad
    monótona. Las líneas con flechas representan distancias entre puntos del
    espacio de características.
    (a) Dominancia: distancia diagonal es mayor que cada una de las distancias
        a lo largo de las dimensiones.
    (b) Consistencia: las relaciones de orden a lo largo de una dimensión son
        independientes de otras dimensiones.
    (c) Transitividad: relaciones de orden traslapadas fuerzan las mismas
        relaciones entre los puntos en la región de traslape.
        los puntos de la región de traslape.
 }
 \label{fig:cons-dom-trans}
\end{figure}


\end{document}
