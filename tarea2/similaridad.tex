% vim: set fileencoding=utf-8 encoding=utf-8:
\documentclass[spanish]{article}
\usepackage{babel}
\usepackage[utf8]{inputenc}
\usepackage{fullpage}
\usepackage{url}
\usepackage{palatino}
\usepackage{mathpazo}
\usepackage{amsmath}
\usepackage{amssymb}
\usepackage{graphicx}


\title{Reconocimiento de Patrones---Discusión del artículo \textit{Similarity~Measures}}
\author{Roberto Bonvallet \\ \url {<rbonvall@gmail.com>}}
\date{Agosto de 2008}

\begin{document}
\maketitle

\section{Resumen}
El artículo \textit{Similarity Measures}~\cite{sim} aborda el problema de medir
disimilaridades en espacios de características, desde el punto de vista de los
requerimientos de un sistema de recuperación de información.


\section{Modelos axiomáticos presentados en el artículo}

\subsection{Axiomas métricos}
\begin{description}
    \item [autosimilaridad constante:]
        $d(A, A) = d(B, B)$;
    \item [minimalidad:]
        $d(A, B)\ge d(A, A)$;
    \item [simetría:]
        $d(A, B) = d(B, A)$;
    \item [desigualdad triangular:]
        $d(A, B) + d(B, C)\ge d(A, C)$.
\end{description}
Muchos modelos de similaridad suponen que el espacio subyacente cumple con estos axiomas,
pero su validez es cuestionada por dos motivos:
\begin{itemize}
    \item experimentos de percepción humana arrojan resultados
        inconsistentes con el modelo euclideano de distancia;
    \item la desigualdad triangular no es verificable experimentalmente,
        pues no se traduce en una propiedad similar para la similaridad juzgada.
\end{itemize}





\subsection{Estructura de proximidad monótona}
La estructura de proximidad monótona es un modelo diseñado para ser verificable experimentalmente.
Esto se logra imponiendo propiedades ordinales, que no se ven afectadas por una función de
generalización monótona.  Este modelo es más general y menos restrictivo que el métrico, y otorga
más importancia al cumplimiento de propiedades a lo largo de los ejes.  Las propiedades son:
\begin{description}
    \item [dominancia:]
        análoga a la desigualdad triangular, pero a lo largo de los ejes;
    \item [consistencia:]
        el orden a lo largo de una dimensión es independiente de otras
        dimensiones;
    \item [transitividad:]
        traslapes de trios ordenados a lo largo de una dimensión inducen un orden más general;
    \item [desigualdad de esquina;]
\end{description}



\subsection{Modelo de contraste de características}
En el modelo de contraste de características~(FCM) los estímulos son representados por un conjunto
de características binarias.
\begin{description}
    \item [calce:] la similaridad tiene la forma $s(a, b) = F(A\cap B, A-B, B-A)$;
    \item [monotonía:] la similaridad está influenciada por el traslape de características comunes;
    \item [independencia:]
\end{description}
Un resultado importante acerca este modelo es el teorema de representación: si una función de
similaridad $s$ satisface las tres propiedades, existe una función de similaridad $S$ de la forma
$S(a, b) = f(A\cap B) - \alpha f(A-B) - \beta f(B-A)$, que es equivalente a $s$ en el sentido que
preserva el orden entre pares de estímulos.  

\subsection{FCM difuso con dependencia de características}
El FCM difuso es una extensión del FCM en que las características binarias (que un estímulo puede
tener o no) son reemplazadas por un conjunto de predicados difusos, cada uno de los cuales el
estímulo satisface en cierta medida $\mu_i$.  Esta extensión permite la incorporación de
características numéricas.

Tras definiciones apropiadas de los operadores difusos de intersección y diferencia de conjuntos,
la función de similaridad de Tversky se define como:
\begin{align}
    S(\phi, \psi) = \sum_i\bigl(\mu_\cap (\phi_i, \psi_i) -
                          \alpha\mu_-  (\phi_i, \psi_i)   -
                          \beta \mu_-  (\psi_i, \phi_i)\bigr)
\end{align}

...




\section{Medidas de similaridad y recuperación de información}
Del artículo es posible rescatar varios criterios que 


\begin{itemize}
    \item recuperar elementos de una base de datos que son descripciones
        incompletas o esbozos de los elementos buscados;
    \item correlación con la similaritud perceptual.
    \item importancia de las características por separado.
    \item relevancia empíricamente observable
    \item comparación consulta-resultado en vez de punto-punto
\end{itemize}



\section{Modelos presentados}
\subsection{Métricos}





% set members for next example
\newcommand{\SMA}{\Box}
\newcommand{\SMB}{\spadesuit}
\newcommand{\SMC}{\heartsuit}
\newcommand{\SMD}{\clubsuit}
\newcommand{\SME}{\diamondsuit}

\begin{align*}
    A &= \{\phantom{\SMA,  \SMB,} \SMC,          \SMD,          \SME\}  \\
    B &= \{\phantom{\SMA,} \SMB,  \SMC,          \SMD\phantom{, \SME}\} \\
    C &=          \{\SMA,  \SMB,  \SMC\phantom{, \SMD,          \SME}\} \\
\end{align*}

\begin{figure}[t]
  \centering
  \includegraphics[bb=0 0 513 166]{imagenes/dom-cons-trans.png}
  % dom-cons-trans.png: 641x208 pixel, 90dpi, 18.09x5.87 cm, bb=0 0 513 166
  \caption{\small %
    Ejemplos de las propiedades de un espacio con estructura de proximidad
    monótona. Las líneas con flechas representan distancias entre puntos del
    espacio de características.
    (a) Dominancia: distancia diagonal es mayor que a lo largo de cada eje.
    (b) Consistencia: las relaciones de orden a lo largo de una dimensión son
        independientes de otras dimensiones.
    (c) Transitividad: relaciones de orden traslapadas fuerzan las mismas
        relaciones entre los puntos en la región de traslape.
  }
  \label{fig:cons-dom-trans}
\end{figure}

\begin{figure}[t]
  \centering
  \includegraphics[bb=0 0 374 125]{imagenes/esquina.png}
  % esquina.png: 467x156 pixel, 90dpi, 13.18x4.40 cm, bb=0 0 374 125
  \label{fig:esquina}
  \caption{\small %
    Comparación entre la desigualdad de esquina y la desigualdad
    triangular.  La desigualdad triangular compara la distancias directa con
    la distancia a lo largo de dimensiones entre dos puntos.  La desigualdad de
    esquina sólo impone estas restricciones por tramos, pero deben cumplirse
    para cualquier lugar de la zona sombreada donde esté el punto central.
  }
\end{figure}




%\begin{thebibliography}{1}
\begin{thebibliography}{99}
    \bibitem{sim}
        Simone Santini, Ramesh Jain.
        \emph{Similarity Measures.}
        IEEE Trans.~on Pattern Analysis and Machine Inteligence.
        Vol.~21, No.~9, September~1999.
    \bibitem{shepard}
        Roger~{}N.~Shepard.
        \emph{Toward a Universal Law of Generalization for Physical Science.}
        Science, vol.~237, pp.~1317--1323, 1987.
\end{thebibliography}

\end{document}
